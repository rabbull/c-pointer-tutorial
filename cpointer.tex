\documentclass{beamer}
\usepackage{xeCJK}
\usepackage{listings}
\usepackage{color}
\usepackage{dirtree}

\setmonofont{Consolas}
\setCJKmainfont{Noto Serif CJK SC}
\setCJKsansfont{Noto Sans CJK SC}
\setCJKmonofont{Noto Sans Mono CJK SC}

\usetheme{Madrid}

\lstdefinestyle{cstyle} {
  language=C,
  keywordstyle=\color{blue},
  stringstyle=\color{violet},
  numberstyle=\color{orange},
  commentstyle=\color{gray},
  frame=shadowbox,
  basicstyle=\ttfamily
}

\author{刘子森}
\title{指针}
\date{\today}
\institute{四川大学软件学院}

\begin{document}

\frame{\titlepage}

\begin{frame}
 \tableofcontents
\end{frame}

\section{指针}

\begin{frame}
 \frametitle{定义}
 指针是用来储存地址的变量。
\end{frame}

\begin{frame}
 \frametitle{用法}
 \lstinputlisting[style=cstyle]{examples/pointer/1/main.c}
\end{frame}

\begin{frame}
 \frametitle{四则运算}
 \lstinputlisting[style=cstyle]{examples/pointer/2/main.c}
\end{frame}

\begin{frame}
 \frametitle{四则运算}
 \begin{itemize}
  \item 指向类型 \(T\) 的指针与整数 \(i\) 的加法在指针的值上加了
        \[offset = i \times sizeof(T)\]
  \item 指针减法相当于其加法中的 \(i\) 取负值
  \item 指针没有乘法和除法
 \end{itemize}
\end{frame}

\section{数组}
\begin{frame}
 \frametitle{又一个栗子}
 \lstinputlisting[style=cstyle]{examples/array/1/main.c}
\end{frame}

\begin{frame}
 \frametitle{又一个栗子}
 \begin{itemize}
  \item 数组的值是一个指向数组首位元素的指针
  \item 数组中相邻两元素地址之差恰是元素大小
 \end{itemize}
\end{frame}

\begin{frame}

 \frametitle{显然}
 \lstinputlisting[style=cstyle]{examples/array/2/main.c}
\end{frame}

\begin{frame}
 \frametitle{显然}

 数组名加下标可以得到该下标对应的元素的地址。

\end{frame}

\begin{frame}
 \frametitle{数组和指针的区别}
 \lstinputlisting[style=cstyle]{examples/array/3/main.c}
\end{frame}

\begin{frame}
 \frametitle{数组和指针的区别}
 \begin{itemize}
  \item 在使用 \(sizeof\) 求大小时两者处理方式不同
  \item 没了
 \end{itemize}
\end{frame}

\begin{frame}
 \frametitle{整活?}
 \lstinputlisting[style=cstyle]{examples/array/4/main.c}
\end{frame}

\section{函数}
\begin{frame}
 \frametitle{函数指针}
 顾名思义:函数指针是指向函数的指针。
 \lstinputlisting[style=cstyle]{examples/function/1/main.c}
\end{frame}

\begin{frame}
 \frametitle{回调函数}

 假设我们实现了一种很厉害的基于比较和交换的排序算法,想要将其推广。我们希望允许用户对任何数据进行排序,包括整数、浮点数甚至结构体。

\end{frame}

\begin{frame}
 \frametitle{回调函数}
 \framesubtitle{给苹果按重量排序}

 调用关系:

 \dirtree{%
  .1 main.
  .2 struct Apple.
  .2 apple\_comp.
  .2 sort.
  .3 swap.
 }
\end{frame}

\begin{frame}
 \frametitle{回调函数}
 \framesubtitle{给苹果按重量排序}

 \lstinputlisting[style=cstyle, linerange={14-27}]{examples/function/2/main.c}
\end{frame}

\begin{frame}
 \frametitle{回调函数}
 \framesubtitle{给苹果按重量排序}

 \lstinputlisting[style=cstyle, linerange={7-13, 28-33}]{examples/function/2/main.c}
\end{frame}

\begin{frame}
 \frametitle{回调函数}
 \framesubtitle{给苹果按重量排序}

 \lstinputlisting[style=cstyle]{examples/function/2/sort.h}
\end{frame}

\begin{frame}
 \frametitle{回调函数}
 \framesubtitle{给苹果按重量排序}

 \lstinputlisting[style=cstyle, linerange={21-31}]{examples/function/2/sort.c}
\end{frame}

\end{document}
